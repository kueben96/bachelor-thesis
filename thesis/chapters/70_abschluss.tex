\section{Kritische Würdigung und Ausblick}

% Zusammenfassung
Mithilfe der Entwicklung eines Prototypen zur Virtualisierung einer simulierten Windenergieanlage wurde in dieser Arbeit die Frage analysiert, wie SAP Leonardo die digitale Transformation in der Energiewirtschaft mit Internet of Things unterstützen kann. Motiviert wurde die Problemstellung von der dezentralen Struktur Energieversorgungsmarktes in Anbetracht der Energiewende. Durch eine umfassende Analyse des Branchenkontextes wurden zur Beantwortung der Forschungsfrage FF1.1 Anforderungen an ein System aus Sicht der dezentralen Energieerzeugung definiert. Als Grundlage für die Anforderungserhebung an das System wurde ein repräsentativer Anwendungsfall für einen Windenergieanlagenhersteller entwickelt. Diese ergab unter anderem, dass die Anlagen und dessen Komponenten eine echtzeitfähige Integration in ein IoT-Netzwerk erfordern, denn:
\\Im Kontext der vierten industriellen Revolution, der \textit{Industrie 4.0}, entsteht eine neuartiges Konzept für Wertschöpfung. Durch die Vernetzung von allen an der Wertschöpfungskette des Produkts beteiligten Instanzen entstehen quantitative Daten, welche durch die Verarbeitung mit intelligenten Diensten qualitative Erkenntnisse bringen. Die Produktionsanlage steht als erstes Glied in der Wertschöpfungskette des Endprodukts Energie. Allein um dieses Endprodukt mithilfe dezentraler Anlagen zu erzeugen, ist eine digitale Infrastruktur nötig. Aber auch um die veränderten Kundenerwartungen im Rahmen von Industrie 4.0 zu erfüllen, muss ein Veränderungsprozess für die digitale Systemlandschaft entstehen. Als Enabler für die Transformation wurde die Nutzung der Technologien wie Cloud Computing, Big Data und Analytics identifiziert. Für die Vereinigung dieser Technologien und die intelligente Vernetzung aller Komponenten im Wertschöpfungsprozess kann eine Innovationsplattform herangezogen werden. Im Rahmen dieser Arbeit wurde angesichts der FF1.2 untersucht, welche Möglichkeiten zur Intelligenten Vernetzung die Systemarchitektur der SAP Leonardo IoT Foundation bietet. 
Hier digitale Transfomation und Veränderungsprozess um Kundenerwartungen zu erfüllen, Neuausrichtung der digitalen Infrastruktur. Cloudbasierte Innovationsplattform -> Um die Anlage im gesamten Wertschöpfungsprozess abbilden zu können wurde eine Innovationsplattform benötigt, welche die Herausforderungen bewältigen kann. Da entsteht FF1.2: Welche Möglichkeiten die Architektur von SAP Leonardo zu intelligenten Vernetzung bietet, wurde in der Systemanalyse erläutert. Eine der wichtigsten Möglichkeiten ist unter anderem die offne Architektur, die mit RAMI Konformität aufweist. Zudem kann man durch SCP Integration die Anlage in eigene Geschäftslogik integrieren aber auch den Kunden (EVU) Dienste zum Überwachen ihres Produkts anbieten. Außerdem Anbindung von intelligenten Diensten wie Machine Learning und Analytics. Schließlich wurde eine Architektur entworfen und umgesetzt. 
Veränderungsprozess in Hinblick auf Kundenerwartungen: digitale Transformation
Schließlich ist eine Architekturvorlage für zukünftige Projekte entstanden. 


% Evaluation und Erfüllung des Ziels % Kritik 
Da es unter der Nutzung von Abstraktionsebenen zur Anforderungserhebung möglich war, eindeutige Schwachstellen des Prototypen zu identifizieren und Handlungsempfehlungen zu geben, ist die Anwendbarkeit dieser Systemarchitektur als validiert und die Zielsetzung dieser Arbeit als erreicht anzusehen.

Jetzt Kritik 
% Ausblick
und Ausblick
Steuerungskomplexität 
Ziel dieser Bachelorarbeit war die Analyse der Frage,  Der Branchenkontext in diesem Fall stel

Problem: dezentralität erfordert Erfassung von Daten und echtzeitfähige Steuerung der Anlagen 
Ziel
Lösung

Frage, ob die derzeitigen Strukturen zeitgemäß sind in Anbetracht der Dynamik der digitalen Welt

Kritik der Lösung: SAP Leo Beta und teilweise unzuverlässig

Anforderungen d. Energiemarktes kurz bennen wegen FF1 und mit FF2 abgleichen. Schwächen des Systems bennen 
Der als Lösung für den Anwendungsfall des Auftraggebers implementierte Prototyp sollte aufzeigen, wie SAP Leonardo die digitale Transformation im Kontext der Energiewirtschaft unterstützen kann. Da eine Implementierung an einer realen Windenergieanlage einen zu großen Kostenfaktor ergeben hätte, sollte die Eignung und Realisierbarkeit anhand einer Simulation geprüft werden. Die Simulation 
Was war das Ziel der Arbeit? Forschungsfragen integrieren
Wie kann die digitale Transformation in der Energiewirtschaft durchgeführt werden

Vieles wird noch nicht unterstützt
Reflexion:
Was hab ich gemacht? (Selbst-Kritisch)und SAP sehr BETA und schlecht dokumentiert blabla
\newline
Ausblick:
Ausblick/Weitere Möglichkeiten
Integration mit SAP Backend
HANA DB
APIs/SDK für Leonardo
Edge Processing mit Interceptors and Adapters, echtes Gerät mit echten Sensorwerten statt RPI und teilweise simulierte Werte
\newline
Beantwortung der Frage:, wie gut man mit SAP Leonardo digital transformieren kann auch nach RAMI 4.0 
Ausblick: Zukunft vernetzte (Industrie) Welt und neues Verständnis von Wertschöpfung


In Anbetracht der erzielten Ergebnisse dieser Ausarbeitung besteht in zweierlei Hinsicht weiterer Forschungsbedarf. 
Entwicklung prozessrelevanter Kennzahlen


\newpage
