\section{Kritische Würdigung und Ausblick}

% Zusammenfassung
Mithilfe der Entwicklung eines Prototypen zur Virtualisierung einer simulierten Windenergieanlage wurde in dieser Arbeit die Frage analysiert, wie SAP Leonardo die digitale Transformation in der Energiewirtschaft mit Internet of Things unterstützen kann. 
\\Motiviert wurde die Problemstellung von der dezentralen Struktur des Energieversorgungsmarktes in Anbetracht der Energiewende. Durch eine umfassende Analyse des Branchenkontextes wurden zur Beantwortung der Forschungsfrage FF1.1 Anforderungen an ein System aus Sicht der dezentralen Energieerzeugung definiert. Als Grundlage für die Anforderungserhebung an das System wurde ein repräsentativer Anwendungsfall für einen Windenergieanlagenhersteller entwickelt. Diese ergab unter anderem, dass die Anlagen und dessen Komponenten eine echtzeitfähige Integration in ein IoT-Netzwerk erfordern, denn:
\\Im Kontext der vierten industriellen Revolution, der \textit{Industrie 4.0}, entsteht ein neuartiges Konzept für Wertschöpfung. Durch die Vernetzung von allen an der Wertschöpfungskette des Produkts beteiligten Instanzen entstehen quantitative Daten, welche durch die Verarbeitung mit intelligenten Diensten qualitative Erkenntnisse bringen. Die Produktionsanlage steht als erstes Glied in der Wertschöpfungskette des Endprodukts Energie. Allein um dieses Endprodukt mithilfe dezentraler Anlagen zu erzeugen, ist aufgrund der Steuerungskomplexität eine digitale Infrastruktur nötig. Aber auch um die veränderten Kundenerwartungen im Rahmen von Industrie 4.0 zu erfüllen, muss ein Veränderungsprozess für die digitale Systemlandschaft entstehen. Als Enabler für die Transformation wurde die Nutzung der Technologien wie Cloud Computing, Big Data und Analytics identifiziert. 
\\Für die Vereinigung dieser Technologien und die intelligente Vernetzung aller Komponenten im Wertschöpfungsprozess kann eine Innovationsplattform herangezogen werden. Im Rahmen dieser Arbeit wurde angesichts der FF1.2 untersucht, welche Möglichkeiten zur intelligenten Vernetzung die Systemarchitektur der SAP Leonardo IoT Foundation bietet. Die Systemanalyse ergab eine offene und flexible Architektur, die Konformität mit der Referenzarchitektur \ac{rami} aufweist. Die Architektur deckt die Anforderungen an eine Industrie-4.0-Lösung ab: Von der vertikalen Integration der physischen Anlage in die Cloud über verschiedene Kommunikationsprotokolle und anschließender Erzeugung eines digitalen Zwillings, bis hin zur horizontalen Integration in den Geschäftskontext durch die Anbindung der SAP Cloud Platform Integration. Intelligent wird die Vernetzung durch die Eingliederung von Diensten wie SAP Leonardo Machine Learning oder Analytics. 
\\ Auf Basis dieser Erkenntnisse wurde gemäß FF1.3 ein System für den Prototyp entworfen, welches die Anforderungen des Ausgangsszenarios deckt. Der Prototyp enthält die Integration einer simulierten Anlage in die \ac{scp}, die anschließende Verarbeitung der Sensordaten in SAP Leonardo IoT, die automatische Interaktion über die Cloud mit der Anlage sowie die Repräsentation des digitalen Zwillings in einer UI5-Anwendung. 
% Kritik
\\\\% Evaluation und Erfüllung des Ziels % Kritik 
Da es unter der Nutzung von Abstraktionsebenen zur Anforderungserhebung möglich war, eindeutige Schwachstellen des Prototyps zu identifizieren und Handlungsempfehlungen zu geben, ist die Anwendbarkeit dieser Systemarchitektur als validiert und die Zielsetzung dieser Arbeit als erreicht anzusehen. 
% Ausblick
\\\\ Fortsetzungsgedarf besteht nun in zweierlei Hinsicht. Einerseits kann auf Grundlage der entworfenen Architekturvorlage eine praktische Lösung angewandt werden. In der Handlungsempfehlung wurden bereits Vorschläge zur Umsetzung ausgeschrieben. Allerdings steckt die SAP-Leonardo-Systemlandschaft, da es sich um ein junges Projekt handelt, noch in der Entwicklungsphase. Viele Funktionen sind noch nicht vollständig ausgereift oder sind mangelhaft dokumentiert. Die Entwicklung mit dem zur Verfügung gestellten Portfolio kann deshalb an Grenzen stoßen. Das Heranziehen eines Experten könnte für eine produktive Nutzung von Nöten sein. Andererseits kann es bei fortgeschrittenen Entwicklungserfahrungen im Bereich cloud-nativer und SAP-basierter Entwicklung durch die zur Verfügung gestellten Microservices viel Flexibilität bieten.  Weitere Vorteile können durch die  Integration von SAP-Stamm- und Bewegungsdaten entstehen. Mit einer vollständigen Integration können ideale Voraussetzungen entstehen, den Kunden intelligente digitale Lösungen und Dienste zur Gewährleistung der Produktqualität anzubieten. 
\\Andererseits könnte die Untersuchung von weiteren Innovationsplattformen, wie z.B. AWS IoT, im Rahmen einer weiteren Forschungsarbeit eventuell Erkenntnisse über eine bessere Eignung bringen. 
\newpage
