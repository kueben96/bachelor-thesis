\section{Kritische Würdigung und Ausblick}

Mithilfe der Entwicklung eines Prototypen zur Virtualisierung einer simulierten Windenergieanlage wurde in dieser Arbeit die Frage analysiert, wie SAP Leonardo die digitale Transformation in der Energiewirtschaft mit Internet of Things unterstützen kann. Motiviert wurde die Problemstellung von der dezentralen Struktur Energieversorgungsmarktes in Anbetracht der Energiewende. Durch eine umfassende Analyse des Branchenkontextes wurden zur Beantwortung der Forschungsfrage FF1.1 Anforderungen an ein System aus Sicht der dezentralen Energieerzeugung definiert. Diese ergab, dass die Anlagen eine echtzeitfähige Integration in ein IoT-Netzwerk erfordern.
Da es unter der Nutzung von Abstraktionsebenen zur Anforderungserhebung möglich war, eindeutige Schwachstellen des Prototypen zu identifizieren und Verbesserungsmaßnahmen zu entwickeln, ist die Anwendbarkeit dieser Systemarchitektur als validiert und die Zielsetzung dieser Arbeit als erreicht anzusehen.
Steuerungskomplexität 
Ziel dieser Bachelorarbeit war die Analyse der Frage,  Der Branchenkontext in diesem Fall stel

Problem: dezentralität erfordert Erfassung von Daten und echtzeitfähige Steuerung der Anlagen 
Ziel
Lösung

Frage, ob die derzeitigen Strukturen zeitgemäß sind in Anbetracht der Dynamik der digitalen Welt

Kritik der Lösung: SAP Leo Beta und teilweise unzuverlässig

Anforderungen d. Energiemarktes kurz bennen wegen FF1 und mit FF2 abgleichen. Schwächen des Systems bennen 
Der als Lösung für den Anwendungsfall des Auftraggebers implementierte Prototyp sollte aufzeigen, wie SAP Leonardo die digitale Transformation im Kontext der Energiewirtschaft unterstützen kann. Da eine Implementierung an einer realen Windenergieanlage einen zu großen Kostenfaktor ergeben hätte, sollte die Eignung und Realisierbarkeit anhand einer Simulation geprüft werden. Die Simulation 
Was war das Ziel der Arbeit? Forschungsfragen integrieren
Wie kann die digitale Transformation in der Energiewirtschaft durchgeführt werden

Vieles wird noch nicht unterstützt
Reflexion:
Was hab ich gemacht? (Selbst-Kritisch) z.B.scheiß-Edge und SAP sehr BETA und schlecht dokumentiert blabla
\newline
Ausblick:
Ausblick/Weitere Möglichkeiten
Integration mit SAP Backend
HANA DB
APIs/SDK für Leonardo
Edge Processing mit Interceptors and Adapters, echtes Gerät mit echten Sensorwerten statt RPI und teilweise simulierte Werte
\newline
Beantwortung der Frage:, wie gut man mit SAP Leonardo digital transformieren kann auch nach RAMI 4.0


In Anbetracht der erzielten Ergebnisse dieser Ausarbeitung besteht in zweierlei Hinsicht weiterer Forschungsbedarf. 
Entwicklung prozessrelevanter Kennzahlen

Ausblick: Zukunft vernetzte (Industrie) Welt und neues Verständnis von Wertschöpfung
\newpage
