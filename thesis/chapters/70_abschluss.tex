\section{Kritische Würdigung und Ausblick}


Anforderungen d. Energiemarktes kurz bennen wegen FF1 und mit FF2 abgleichen. Schwächen des Systems bennen 
Der als Lösung für den Anwendungsfall des Auftraggebers implementierte Prototyp sollte aufzeigen, wie SAP Leonardo die digitale Transformation im Kontext der Energiewirtschaft unterstützen kann. Da eine Implementierung an einer realen Windenergieanlage einen zu großen Kostenfaktor ergeben hätte, sollte die Eignung und Realisierbarkeit anhand einer Simulation geprüft werden. Die Simulation 
Was war das Ziel der Arbeit? Forschungsfragen integrieren
Wie kann die digitale Transformation in der Energiewirtschaft durchgeführt werden

Vieles wird noch nicht unterstützt
Reflexion:
Was hab ich gemacht? (Selbst-Kritisch) z.B.scheiß-Edge und SAP sehr BETA und schlecht dokumentiert blabla
\newline
Ausblick:
Ausblick/Weitere Möglichkeiten
Integration mit SAP Backend
HANA DB
APIs/SDK für Leonardo
Edge Processing mit Interceptors and Adapters, echtes Gerät mit echten Sensorwerten statt RPI und teilweise simulierte Werte
\newline
Beantwortung der Frage:, wie gut man mit SAP Leonardo digital transformieren kann auch nach RAMI 4.0

Ausblick: Zukunft vernetzte (Industrie) Welt und neues Verständnis von Wertschöpfung
\newpage
