\subsection{Toolset/Innovationsplattformen/Werkzeuge}
In den letzten 45 Jahren haben sich die SAP-Technologien durch kontinuierliche Veränderungen an die Anforderungen der digitalen Welt angepasst. Die Anfänge der Datenverarbeitung von SAP basierte in den 1960er Jahren auf lokalen PCs und der Mainframe-Architektur. Mit der Client-Server-Architektur und dem darauf basierenden R/3-System konnte die Software ab den 1990er Jahren eine größere Vernetzung und somit einen größeren Informationsaustausch ermöglichen. Mit der Verbreitung des Internets und dem Ausbau des mobilen Breitbandnetzes begann zwischen 2000 und 2010 mit den Technologien wie Cloud, Mobile und Big Data eine digitale Transformation \citep[S. 44]{Elsner2018}. Mit der rasant wachsenden Datenmenge entstanden Möglichkeiten, diese intelligent zu vernetzen und einen Mehrwert daraus zu schöpfen.

Im folgenden Kapitel werden ausgewählte Plattformen und deren Eigenschaften beschrieben.

\subsubsection{SAP Cloud Platform}

Die SAP Cloud Platform ist eine offene PaaS, die auch SaaS für Kunden bereitstellt.

SAP Cloud Platform und AWS Microservices und APIs
Programmiersprachen und Laufzeitumgebungen
alle gängigen IaaS können zum deployen genutzt werden
CF, NEO, ABAP
Destinationsss
Innovationsplattform
Functional Services und Business Services
\subsubsection{SAP Leonardo}

Die SAP Cloud Platform bildet die technologische Grundlage für die digitale Transformation mit SAP. Diese
Technologien werden um die \textit{SAP-Leonardo-Technologien} ergänzt. Der Begriff lehnt sich an den Innovationsgeist Leonardo da Vincis an und soll die \textit{digitale Rennaissance} mit SAP  darstellen \citep{Howells2017}.
SAP Leonardo ist nicht als Plattform oder Produkt zu verstehen. Es handelt sich vielmehr um eine Sammlung von Functional Services (s. 2.2.3 Cloud Computing), die als zukunftsgestaltende Technologien gesehen werden \citep{Elsner2018}. Zu diesen Technologien
gehören neben dem Internet der Dinge (\acf{iot}) auch Machine Learning, Big Data, Blockchain und Advanced Analytics.
Auch hierbei gilt das Prinzip des \textcolor{red}{Netzwerkeffekts} (s. 2.1.4): der Mehrwert besteht in der integrierten Anwendung in einem Kernsystem \citep{Elsner2018}.

GUI, API, SDKs

\subsubsection{AWS Cloud}
SNS-Server
Außerdem basiert SCP auf AWS

\subsubsection{Git}

\newpage
