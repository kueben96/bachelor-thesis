\subsection{Toolset/Innovationsplattformen/Werkzeuge} \label{toolset}

Die SAP Cloud Platform bildet

In den letzten 45 Jahren haben sich die SAP-Technologien durch kontinuierliche Veränderungen an die Anforderungen der digitalen Welt angepasst. Die Anfänge der Datenverarbeitung von SAP basierte in den 1960er Jahren auf lokalen PCs und der Mainframe-Architektur. Mit der Client-Server-Architektur und dem darauf basierenden R/3-System konnte die Software ab den 1990er Jahren eine größere Vernetzung und somit einen größeren Informationsaustausch ermöglichen. Mit der Verbreitung des Internets und dem Ausbau des mobilen Breitbandnetzes begann zwischen 2000 und 2010 mit den Technologien wie Cloud, Mobile und Big Data eine digitale Transformation \citep[S. 44]{Elsner2018}. Mit der rasant wachsenden Datenmenge entstanden Möglichkeiten, diese intelligent zu vernetzen und einen Mehrwert daraus zu schöpfen.
SAP Scloud Platform bildet die technologische Basis für die digitale Transformation mit SAP.
HANA-Anwendungen auf Microservice-Ebene für die Verarbeitung von Massendaten

Die bereits mehrfach erwähnten Cloud-Dienste: Erwähnenen, dass IaaS, PaaS notwendig ist und evtl auch SaaS in SCP
Im folgenden Kapitel werden ausgewählte Plattformen und deren Eigenschaften beschrieben.

\subsubsection{SAP Cloud Platform} \label{scp}

Die \acf{scp} ist eine offene \ac{paas}, die eine Entwicklungs- und Laufzeitumgebung für cloud-native Anwendungen bereitstellt und das Rückgrat der Services des SAP Leonardo Portfolios bildet. Unterstützt wird der Betrieb von verschiedenen Global Playern unter den Infrastrukturanbietern wie \ac{aws}, \ac{gcp}, SAP oder Microsoft Azure. Im Grunde ermöglicht die Cloud Plattform drei Dinge. Zum einen können Anwendungen zur Erweiterung von On-Premise-Lösungen oder Cloud-Lösungen entwickelt werden. Zudem bietet \ac{scp} Integrationsservices an, um Cloud-Dienste in das SAP-Backend aber auch in heterogene Systemquellen zu integrieren. Anwendungen können aber auch mithilfe von Tools und Microservices von Grund auf entwickelt werden, ohne Hardware und Laufzeit einrichten zu müssen \citep{Acharya2019}.
Wenn von \textit{Services} gesprochen wird, unterscheidet man zwischen zwei Arten.
Die \textit{Functional Services} werden genutzt, um Anwendungen nach eigenen Anforderungen zu entwickeln. Es werden z.B. Services aus den Bereichen User-Interface-Entwicklung, mobile Anwendungen, Sicherheit und Authentifizierung, Datenbankdienste und Integration angeboten \citep{Elsner2018}. Wenn die Services von SAP identifizierte Anwendungsfälle behandeln, werden sie als \textit{Business Services} bezeichnet. Diese Dienste müssen für den Einsatz nur noch um den individuellen Geschäftskontext erweitert werden \citep{Utecht2018}.

\noindent Wie die Anwendungen und Services entwickelt und verwaltet werden können, hängt von der Umgebung ab. Die \ac{scp} stellt drei Umgebungen zur Verfügung, von denen jede mindestens eine Laufzeitumgebung sowie verschiedene Tools zur Entwicklung bereitstellt: Neo, Cloud Foundry und ABAP.

\paragraph{Neo-Environment} Die SAP Cloud Platform wurde 2012 erstmalig nur in der Neo-Umgebung angeboten und unterstützte nur die Entwicklung von Java-, nativen HANA- und HTML5-Anwendungen \citep{Elsner2018}.

\paragraph{Cloud Foundry-Environment} 2016 wurde die Beta-Version der Open-Source Multi-Cloud-PaaS von VMWare und General Electric entwickelt \citep{Utecht2018}. Schließlich haben sich viele große Technolgieunternehmen wie IBM, Pivotal, Google oder SAP sich für eine gemeinsame Laufzeitumgebung entschieden. Somit wurden sie Teil der Non-Profit-Initiative \textit{Cloud Foundry Foundation}, um die Technologie weiterzuentwickeln. Durch die gemeinsame Basis wird die Interoperabilität von verschiedensten Progammiersprachen und Applikationsdiensten gewährleistet (s. Anhang \ref{cf_table}).  Deshalb ergänzte SAP die \ac{scp} um die Cloud Foundry-Umgebung, in der die Nutzung von Services für das Internet der Dinge, Blockchain oder Machine Learning möglich ist \citep{Elsner2018}. Aufgrund der Kollaboration und den daraus entstandenen Möglichkeiten ist Cloud Foundry mittlerweile ein industrieller Standard geworden \citep{Acharya2019}.

\subsubsection{SAP Leonardo}

Die SAP Cloud Platform bildet die technologische Grundlage für die digitale Transformation mit SAP. Diese Technologien werden um die \textit{SAP-Leonardo-Technologien} ergänzt. Der Begriff lehnt sich an den Innovationsgeist Leonardo da Vincis an und soll die \textit{digitale Rennaissance} mit SAP  darstellen \citep{Howells2017}. Es gilt zu betonen, dass SAP Leonardo nicht als Plattform oder Produkt zu verstehen ist. Es handelt sich vielmehr um eine Sammlung von Functional Services (Vgl.\ref{cloud} Microservices) und Anwendungen der \ac{scp}, die als zukunftsgestaltende Technologien gelten \citep{Elsner2018}. Zu diesen Technologien gehören neben dem Internet der Dinge (\acf{iot}) auch Machine Learning, Big Data, Blockchain und Advanced Analytics. Wie bereits zuvor erwähnt, sind diese Technologien in ihrem Wesen nicht neu. Der innovative Charakter entsteht erst durch die Fähigkeit, die zusammenhängenden Technologien miteinander zu verknüpfen \citep{Utecht2018}. Für den Zweck bietet SAP vierlei Lösungen:
 \\Einerseits haben Kunden die Möglichkeit, fertige Lösungen \textit{Powered by SAP Leonardo} zu erwerben, welche nur noch an die Anforderungen des Unternehmens angepasst werden müssen \citep{Utecht2018}. Beispiele für solche Anwendungen sind \textit{Connected Goods}, \textit{Connected Assets} oder \textit{Connected Infrastructure} \citep{Elsner2018}. Mit der Cloud-Anwendung \textit{SAP Leonardo Bridge} können die Daten aus diesen \ac{iot}-Anwendungen mit den Backend-Daten zugesammengeführt, dargestellt und verarbeitet werden. Das Produkt \ac{iot}-Edge ist die Schnittstelle zwischen den Sensoren am Gerät und der Cloud. Bevor die Daten an die Cloud übertragen werden, können sie im Offline-Betrieb gepuffert, aggregiert und vorverarbeitet werden \citep{Utecht2018}.
\\Allerdings können \ac{iot}-Szenarien mit der \textit{SAP Leonardo IoT Foundation} agil selbst entwickelt werden \citep{Elsner2018}. Hierfür ist eine Sammlung von Microservices und \ac{api} auf der \ac{scp} bereitgestellt. Diese können für das \textit{Thing Management} und \textit{SAP IoT Application Enablement} verwendet werden. Im \textit{Thing Management} werden die Geräte nach einem vorgedachten Modell verwaltet, für die im \textit{Application Enablement} ein \textit{digitaler Zwilling} als Grundlage für die Anwendungsentwicklung erstellt wird \citep{Elsner2018}.
\\Im Grunde ist SAP Leonardo ein Sammelbegriff für alle \ac{iot}-relevanten Produkte \citep{Utecht2018}. Kennzeichnend für die Innovation ist neben der Interoperabilität der Services die Interoperabilität der Unternehmensdaten. Das Unternehmensgedächtnis ist in dem digitalen Kern, also dem \ac{erp}-System, verankert \citep{Elsner2018}. Der digitale Kern kann nahtlos mit der agilen Entwicklung auf dem \textit{Digital Innovation System} verbunden werden. Mithilfe  z.B von SAP zur Verfügung gestellten Design Thinking Methoden können aus kontextbezogenen Stamm- und Historiendaten des Unternehmens Erkenntnisse für Geschäftsprozesse geschlossen werden \citep{Elsner2018}.
\\SAP spricht von einem Ökosystem, in dem Entwicklungs- und Implementierungspartner mit den Technologien interagieren, um innovative und effiziente Produkte zu erzeugen. Während Implementierungspartner dabei helfen, bereits bestehende Lösungen und Anwendungen zu konfigurieren und anzupassen, entwickeln die Entwicklungspartner auf Basis der SAP-Leonardo-Technologien neue Anwendungen. Dies ist vor allem im Kontext der Verschmelzung der IT mit dem Maschinenbau und der Elektrotechnik interessant. Die Konvergenz der Themenfelder bewegt reine Hardwarehersteller zu der Entwicklung von intelligenten Softwarelösungen als digitale Service Provider \citep{Elsner2018}. Das Ökosystem beinhaltet zudem die in Abschnitt \ref{scp} erwähnten Infrastrukturpartner, Softwaredienste und Laufzeitumgebungen aber auch z.B. Standards und Normen.


\subsubsection{Amazon Web Services}

\acf{aws} ist ein Cloud-Computing Anbieter aus den USA und ist eine Tochtergesellschaft des Online-Versandhändlers Amazon. Mit 35 \% Marktanteil ist das Unternehmen international führender Cloud-Dienstleister sowohl für Unternehmen als auch für Privatpersonen \citep{awsms2019}. Mit Rechenzentren in den USA, Europa, Brasilien, Asien und Australien stellt es die größte Public-Cloud der Welt zur Verfügung. Zu den Dienstleistungen gehören unter anderem die \ac{paas} Elastic Beanstalk als Entwicklungsumgebung, Anwendungen, Datenbank- und Speicherdienste sowie Netzwerke. Im AWS Marketplace können sowohl die Services von AWS als auch von Dritten bereitgstellte Services gebucht werden. Einige bekannte Dienste von AWS sind der \acf{sns} oder der Simple Workflow Service. In erster Linie stellt AWS die Infrastruktur und Rechenleistung zur Verfügung, die z.B. auch nur für das Hosten eigener Websites genutzt werden kann \citep{AWS}. Das Unternehmen stellt jedoch auch die Infrastuktur für \ac{paas} anderer großer Firmen wie SAP, mit denen seit 2011 eine Infrastrukturpartnerschaft besteht \citep{Elsner2018}. Sowohl die SAP Cloud Plattform sowie einzelne Services wie die der SAP Leonardo IoT Foundation laufen standardmäßig auf Rechenzentren von AWS. Weitere bekannte Großkunden sind Netflix, die NASA oder Dropbox \citep{awsms2019}.



\newpage
