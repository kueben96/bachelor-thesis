\subsection{Toolset/Innovationsplattformen}
In den letzten 45 Jahren haben sich die SAP-Technologien durch kontinuierliche Veränderungen an die Anforderungen der digitalen Welt angepasst.
Die Anfänge der Datenverarbeitung von SAP basierte in den 1960er Jahren auf lokalen PCs und der Mainframe-Architektur.
Mit der Client-Server-Architektur und dem darauf basierenden R/3-System konnte die Software ab den 1990er Jahren eine größere
Vernetzung und somit einen größeren Informationsaustausch ermöglichen. Mit der Verbreitung des Internets und dem Ausbau des mobilen
Breitbandnetzes begann zwischen 2000 und 2010 mit den Technologien wie Cloud, Mobile und Big Data eine digitale Transformation \citep[S. 44]{Elsner2018}.
Mit der rasant wachsenden Datenmenge entstanden Möglichkeiten, diese intelligent zu vernetzen und einen Mehrwert daraus zu schöpfen.

\par Im folgenden Kapitel werden ausgewählte Plattformen und deren Eigenschaften beschrieben.
\subsubsection{AWS Cloud}
SNS-Server
Außerdem basiert SCP auf AWS
\subsubsection{SAP Cloud Platform}

SAP Cloud Platform und AWS Microservices und APIs
Programmiersprachen und Laufzeitumgebungen
CF, NEO, ABAP
Destinations

\subsubsection{SAP Leonardo}

Innovationsplattform
GUI, API, SDKs



\newpage
