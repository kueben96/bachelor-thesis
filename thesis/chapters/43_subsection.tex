\subsection{Toolset/Innovationsplattformen/Werkzeuge} \label{toolset}

In den letzten 45 Jahren haben sich die SAP-Technologien durch kontinuierliche Veränderungen an die Anforderungen der digitalen Welt angepasst. Die Anfänge der Datenverarbeitung von SAP basierte in den 1960er Jahren auf lokalen PCs und der Mainframe-Architektur. Mit der Client-Server-Architektur und dem darauf basierenden R/3-System konnte die Software ab den 1990er Jahren eine größere Vernetzung und somit einen größeren Informationsaustausch ermöglichen. Mit der Verbreitung des Internets und dem Ausbau des mobilen Breitbandnetzes begann zwischen 2000 und 2010 mit den Technologien wie Cloud, Mobile und Big Data eine digitale Transformation \citep[S. 44]{Elsner2018}. Mit der rasant wachsenden Datenmenge entstanden Möglichkeiten, diese intelligent zu vernetzen und einen Mehrwert daraus zu schöpfen.
Auch die Datenverarbeitung von SAP
HANA-Anwendungen auf Microservice-Ebene für die Verarbeitung von Massendaten

Die bereits mehrfach erwähnten Cloud-Dienste: Erwähnenen, dass IaaS, PaaS notwendig ist und evtl auch SaaS in SCP
Im folgenden Kapitel werden ausgewählte Plattformen und deren Eigenschaften beschrieben.

\subsubsection{SAP Cloud Platform} \label{scp}

Die SAP Cloud Platform ist eine offene PaaS, die auch SaaS für Kunden bereitstellt.

SAP Cloud Platform und AWS Microservices und APIs
Programmiersprachen und Laufzeitumgebungen
alle gängigen IaaS können zum deployen genutzt werden
Entwicklungsumgebung SAP Web IDE und Laufzeitumgebungen
CF, NEO, ABAP
Destinationsss
Innovationsplattform
Functional Services und Business Services
Cloud-to-Cloud Integration möglich (Elsner S.247) um Kompetenzen zu vereinen. z.B. Telekom Cloud mit SCP
Embedded Platform as a Service mit zwei Lösungsvarianten. SAP Cloud Platform - Neo Environment and SAP Cloud Platform - Cloud Foundry Environment
Datenbankservices wie verschiedene SAP-HANA-Versionen, MongoDB, PostgreSQL usw
Portalservices: Fiori Launchpad als Anwendungsportal
Application Runtime als Java-Anwendung oder XS-Anwendung Entwicklung mit WEB IDE
Konnektivitätsservice: SAP Cloud Platform Integration
\ac{api}

\begin{quotation}
  cf push - and your app is alive -> in die Systemanalyse
\end{quotation}
\subsubsection{SAP Leonardo}

Die SAP Cloud Platform bildet die technologische Grundlage für die digitale Transformation mit SAP. Diese Technologien werden um die \textit{SAP-Leonardo-Technologien} ergänzt. Der Begriff lehnt sich an den Innovationsgeist Leonardo da Vincis an und soll die \textit{digitale Rennaissance} mit SAP  darstellen \citep{Howells2017}. Es gilt zu betonen, dass SAP Leonardo nicht als Plattform oder Produkt zu verstehen ist. Es handelt sich vielmehr um eine Sammlung von Functional Services (Vgl.\ref{cloud} Microservices) und Anwendungen der \ac{scp}, die als zukunftsgestaltende Technologien gelten \citep{Elsner2018}. Zu diesen Technologien gehören neben dem Internet der Dinge (\acf{iot}) auch Machine Learning, Big Data, Blockchain und Advanced Analytics. Wie bereits zuvor erwähnt, sind diese Technologien in ihrem Wesen nicht neu. Der innovative Charakter entsteht erst durch die Fähigkeit, die zusammenhängenden Technologien miteinander zu verknüpfen \citep{Utecht2018}. Für den Zweck bietet SAP vierlei Lösungen:
 \\\\Einerseits haben Kunden die Möglichkeit, fertige Lösungen \textit{Powered by SAP Leonardo} zu erwerben, welche nur noch an die Anforderungen des Unternehmens angepasst werden müssen \citep{Utecht2018}. Beispiele für solche Anwendungen sind \textit{Connected Goods}, \textit{Connected Assets} oder \textit{Connected Infrastructure} \citep{Elsner2018}. Mit der Cloud-Anwendung \textit{SAP Leonardo Bridge} können die Daten aus diesen \ac{iot}-Anwendungen mit den Backend-Daten zugesammengeführt, dargestellt und verarbeitet werden. Das Produkt \ac{iot}-Edge ist die Schnittstelle zwischen den Sensoren am Gerät und der Cloud. Bevor die Daten an die Cloud übertragen werden, können sie im Offline-Betrieb gepuffert, aggregiert und vorverarbeitet werden \citep{Utecht2018}.
\\Allerdings können \ac{iot}-Szenarien mit der \textit{SAP Leonardo IoT Foundation} agil selbst entwickelt werden \citep{Elsner2018}. Hierfür ist eine Sammlung von Microservices und \ac{api}s auf der \ac{scp} bereitgestellt. Diese können für das \textit{Thing Management} und \textit{SAP IoT Application Enablement} verwendet werden. Im \textit{Thing Management} werden die Geräte nach einem vorgedachten Modell verwaltet, für die im \textit{Application Enablement} ein \textit{digitaler Zwilling} als Grundlage für die Anwendungsentwickelt erstellt wird \citep{Elsner2018}.
\\Im Grunde ist SAP Leonardo ein Sammelbegriff für alle \ac{iot}-relevanten Produkte \citep{Utecht2018}. \textcolor{red}{digitaler Kern als Gedächtnis}
SAP spricht von einem Ökosystem, in dem Entwicklungs- und Implementierungspartner mit den Technologien interagieren, um innovative und effiziente Produkte zu erzeugen. Während Implementierungspartner dabei helfen, bereits bestehende Lösungen und Anwendungen zu konfigurieren und anzupassen, entwickeln die Entwicklungspartner auf Basis der SAP-Leonardo-Technologien neue Anwendungen. Dies ist vor allem im Kontext der Verschmelzung der IT mit dem Maschinenbau und der Elektrotechnik interessant. Die Konvergenz der Themenfelder bewegt reine Hardwarehersteller zu der Entwicklung von intelligenten Softwarelösungen als digitale Service Provider \citep{Elsner2018}. Das Ökosystem beinhaltet zudem die in Abschnitt \ref{scp} erwähnten Infrastrukturpartner, Softwaredienste und Laufzeitumgebungen aber auch z.B. Standards und Normen.
\\\\Auch hierbei gilt das Prinzip des \textcolor{red}{Netzwerkeffekts} (s. 2.1.4): der Mehrwert besteht in der integrierten Anwendung in einem Kernsystem \citep{Elsner2018}.


Vorgedachtes Modell für Geräte (Thing Modell), IoT Services Anbindung Sensoren und Endgeräten, kalter, warmer, heißer Speicher
Best Practices and Design Thinking
\begin{itemize}
  \item Innovationsplattform ist nahtlos mit dem digitalen Kern (ERP) als Unternehmensgedächtnis \citep{Utecht2018} verbunden -> bimodale IT
  \item Einfache Anbindung an Backend Systeme
  \item z.B. ohne Materialstamm und Equipmentstamm kein Digitaler Zwlling
  \item Technologien IoT, Machine Learning, Blockchain, Analytics auch kurz beschreiben, was SAP da anbietet
  \item Auch eigene Leonardo-Produkte wie Connected Goods cloudbasierte IoT Anwendung, für Moni†oring von Produkten um ihren Wert zu maximieren wie Kühlketten, Verbräuche und Bestände
  \item SAP Connected Assets für Anlagegüter
\end{itemize}




Leonardo erlaub die Wiederholung des Design Thinking Prozesses, da mit SCP und Microservice architekur easy skalierbar und anpassbar
Das Bild vielleicht eher in Umsetzungskonzept? oder in Vorgehen


\subsubsection{AWS Cloud}
SNS-Server
Außerdem basiert SCP auf AWS

\subsubsection{Git}

\newpage
