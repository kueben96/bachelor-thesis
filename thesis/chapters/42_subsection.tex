\subsection{Digitale Transformation mit Internet of Things}
Überschriften werden in \LaTeX{} mit den Befehlen $\backslash$section\{\}, $\backslash$subsection\{\} und $\backslash$paragraph\{\} erezugt.

Jeder Überschrift sollte auf der tiefsten Gliederungsebene mindestens eine Seite Text folgen, davon mindestens zwei Zeilen auf derselben Seite. Es sollten nicht mehr als vier Gliederungsebenen verwendet werden.

Überschriften sollten in eine Zeile passen, damit Silbentrennungen vermieden werden können. Sollten Silbentrennungen in Ausnahmefällen erforderlich sein, ist sinngemäß zu trennen, also z.B. nicht Umweltin-formatik, sondern Umwelt-informatik.

\subsubsection{Anforderungen an ein IoT-System}
nach RAMI 4.0
RAMI (Referenzarchitekturmodell Industrie) 4.0: OPC-UA: Kommunikationsstandards (inkl. Sicherheit)
Sensorik: Bedeutung und sehr oberflächlich Funktionsweisen beschreiben
Gateways: Edge Processing
Device Management
Digital Twins

\subsubsection{Cloud Computing}
Cloud-Native-Development -> neues Paradigma fernab vom 3-Thier Modell
IaaS, PaaS, SaaS, Microservices/SOA: service-oriented architecture
Integration modularer Services!= monolithische Strukur
Integration heterogener Datenquellen
