\section{Evaluation}

In diesem Kapitel wird untersucht, ob und wie der umgesetzte Prototyp den Anforderungen an das System gerecht werden kann. Ähnlich wie die Anforderungsanalyse, erfolgt auch die Evaluation für die Abstraktionsebenen Kontextebene, Systemebene und technologische Ebene. Auf Basis dieser Evaluationen wird schließlich eine passende Handlungsempfehlung für den Anwendungsfall gegeben. 

% Kontextebene
\subsection{Evaluation auf Kontextebene}
In der Kontextebene wurden Anforderungen an das System gestellt, welche sich aus den Einflussfaktoren im Systemkontext ergeben. Besonders die Probleme im Zusammenhang mit der Branche der Energiewirtschaft bilden die Kernanforderungen an das System (s. Anhang \ref{anfkontext}). Die sich daraus ergebenden funktionalen Anforderungen (K-FA-1) können im Großen und Ganzen als erfüllt betrachtet werden. Angefordert war es, dem Nutzer die Einsicht auf den digitalen Zwilling einer Anlage zu gewähren, einschließlich der Anzeige von Messwerten, Standorten und prädiktiven Informationen. Mit Betrachtung der Abbildungen \ref{thingpage}, \ref{landing}, \ref{detailoverview} und \ref{thingdetail} scheinen die Anforderungen erfüllt: Die Anlage wird auf der Startseite auf einer Landkarte verortet, die Übersicht über den Zustand liefert Bewertungen des Zustands und die Detailansicht listet einzelne Messwerte auf. Anhand der Einbindung des \ac{sns} von \ac{aws}, aber auch des Befehls zum Aufleuchten der LED, kann dem Wartungspersonal die Reaktion auf kritische Zustände ermöglicht werden. Probleme entstehen bei der Betrachtung der Anforderung \textit{Echtzeit}. Wie in Abbildung \ref{datavisual} dargestellt, werden die Daten sofort an die Cloud transferiert. Allerdings benötigt die Reaktion auf die Daten, also die Aktivierung der HTTP-Anfragen durch die Aktionen, eine Zeitspanne zwischen 20 Sekunden und 2 Minuten. Diese Zeitspanne ist zwar immer noch kürzer als der 10"=Minuten"=Takt des \ac{scada}-Systems, erfüllt aber nicht die Anforderung \textit{Echtzeit}. Nichtsdestotrotz werden die definierten Anwendungsfälle durch den Prototypen realisiert. 
Auch die qualitativen Anforderungen können größtenteils als erfüllt betrachtet werden. Die \ac{soa} von SAP Leonardo und der SAP Cloud Platform liefert die Flexibilität, das System an Veränderungen anzupassen, neue Funktionen einzubinden und neue Geräte hinzuzufügen. In diesem Prototypen wurde dies anhand von Destinationen realisiert. Zudem können über Schnittstellen weitere intelligente Dienste von SAP Leonardo und andere Funktionen (s. Abbildung \ref{leoae}) angebunden werden. Dass die Randbedingungen (K-RA-2 und K-RA-3) zur Orientierung an der \ac{rami} erfüllt werden, wird nach der Systemanalyse deutlich. Die Architektur von SAP Leonardo wird auf die Schichten der IT-Sicht von \ac{rami} angewandt. Das Ergebnis wird in Abbildung \ref{ramicustom} dargestellt. Allerdings wird in der Umsetzung der Kommunikation im Prototypen nicht, wie empfohlen, das OPC-UA-Protokoll verwendet. 

% Systemebene
\subsection{Evaluation auf Systemebene}

Wie das System die Anforderungen aus der Kontextebene erfüllen soll, wurde in der Systemebene definiert. Der innerere logische Aufbau des umgesetzten Prototypen erfüllt mit einigen Schwachstellen die Anforderungen aus Anhang \ref{anf_system}.

Message Processing lahm: Übertragung der Daten an digitalen Zwilling
Dass Echtzeit nicht immer erfüllt wird, deuetet auf Unzuverlässigkeit hin.

% technologische Ebene
\subsection{Evaluation auf technologischer Ebene}


\subsection{Handlungsempfehlung}
Anbindung der realen Anlagen über das OPC-UA Protokoll mit der Gateway Edge
Anbindung an HANA Datenbank, da schnellere In-Memrory Verarbeitung als PostGreSQL
Forschungsfragen
Anforderungen
RAMI 


Komplett Anforderungsanalyse abgleichen
Unflexible UI-Entwicklung weil Module und Packages nicht dokumentiert
OPC UA in Edge Möglich aber im Prototypen nicht implementiert

Abbildung \ref{datavisual} kann entnommen werden, dass die Daten zuverlässlig im 5-Sekunden-Takt empfangen werden. Problematisch ist die Verarbeitung in der Leo-Umgebung.
API Reaktionszeit schlecht und fehleranfällig und unzuverlässige Datenverarbeitung im SAP Leonardo IoT Regelwerk,
Regeln werden manchmal einfach nicht ausgelöst.


Das System bietet sicherlich mehr Möglichkeiten und ist bestimmt stabiler. Trotz umfassender API-Dokumentation ist der Nutzer des Systems nicht ausreichend mit Anwenderdokumentaion versorgt. Hilfreich sind die Blog-Beiträge der SAP-Community, welche jedoch nur anwendunfsfallbasierte Informationen liefern.
\newpage
