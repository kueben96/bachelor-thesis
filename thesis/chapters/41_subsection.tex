\subsection{Industrie 4.0}
In diesem Kapitel Thematik Industrie 4.0 gesellschaftlicher/wirtschaftlicher und technischer Kontext
\subsubsection{Definition}

Laut der \cite{FraunhoferGesellschaft2016} habe die \textit{Industrie 4.0} einen revolutionären Einfluss auf die Wertschöpfung in der Industrie und somit auf die Volkswirtschaft. Dieser Marketingbegriff prägt heute die Agenda vieler Unternehmen und Forschungseinrichtungen. Doch was genau hinter dem Begriff zu verstehen ist, bleibt aufgrund des Fehlens einer  \glqq wissenschaftlichen Präzision\grqq{} uneindeutig \citep{Bendel2019}. Für die Gestaltung der digitalen Transformation entstand das Netzwerk \textit{Plattform Industrie 4.0} zwischen der Bundesregierung, Forschungseinrichtungen und Wirtschaft. Dieses hat zum Ziel, die Produktion mittels modernster Informations- und Kommunikationstechnologien entlang der Wertschöpfungkette \glqq flexibler, individueller und effizienter\grqq{} gestalten \citep{BWE2019}. In der Umsetzungstrategie der BITKOM e.V wird der Begriff wie folgt definiert:

\begin{quotation} \glqq Der Begriff Industrie 4.0 steht für die vierte industrielle Revolution, einer neuen Stufe der Organisation und Steuerung der gesamten Wertschöpfungskette über den Lebenszyklus von Produkten. Dieser Zyklus orientiert an den zunehmend individualisierten Kundenwünschen und erstreckt sich von der Idee, dem Auftrag über die Entwicklung und Fertigung, die Auslieferung eines Produkts an den Endkunden bis hin zum Recycling einschließlich der damit verbundenen Diensleistungen.\grqq{} \citep[S. 8]{BITKOM2015}
\end{quotation}

\subsubsection{Historischer Kontext}
\citep{Barthelmaes2017}

\subsubsection{Technologische Treiber}
Blockchain, Machine Learning, Big Data, Internet of Things, Ubiquitous Computing, Cloud Computing (kurz erwähnen und beschreiben)
\subsubsection{Kommunikationssysteme}
Kosten/Nutzen von Kommunikationssytemen
Metcalfe’s Law, Gilder’s Law, Moore’s Law
\begin{itemize}
  \item Kommunikationssprotokolle und Standards
  \item MQTT
  \item REST
  \item OPCUA
  \item etc
  \item nicht neue subsections sondern einfach Paragraph
\end{itemize}
