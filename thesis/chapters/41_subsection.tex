\subsection{Industrie 4.0}
Dieses Kapitel soll die Relevanz der Thematik verdeutlichen, indem es in einen gesellschaftlichen und wirtschaftlichen Kontext gebracht wird. Zunächst wird der Begriff der \textit{Industrie 4.0}
In diesem Kapitel Thematik Industrie 4.0 gesellschaftlicher/wirtschaftlicher und technischer Kontext
\subsubsection{Definition}

Laut der \cite{FraunhoferGesellschaft2016} habe die \textit{Industrie 4.0} einen revolutionären Einfluss auf die Wertschöpfung in der Industrie und somit auf die Volkswirtschaft. Dieser Marketingbegriff prägt heute die Agenda vieler Unternehmen und Forschungseinrichtungen. Doch was genau hinter dem Begriff zu verstehen ist, bleibt aufgrund des Fehlens einer  \glqq wissenschaftlichen Präzision\grqq{} uneindeutig \citep{Bendel2019}. Für die Gestaltung der digitalen Transformation entstand das Netzwerk \textit{Plattform Industrie 4.0} zwischen der Bundesregierung, Forschungseinrichtungen und Wirtschaft. Dieses hat zum Ziel, die Produktion mittels modernster Informations- und Kommunikationstechnologien entlang der Wertschöpfungkette \glqq flexibler, individueller und effizienter\grqq{} gestalten \citep{BWE2019}. In der Umsetzungstrategie der \citet[S. 8]{BITKOM2015} wird der Begriff wie folgt definiert:

\begin{quotation} \glqq Der Begriff Industrie 4.0 steht für die vierte industrielle Revolution, einer neuen Stufe der Organisation und Steuerung der gesamten Wertschöpfungskette über den Lebenszyklus von Produkten. Dieser Zyklus orientiert an den zunehmend individualisierten Kundenwünschen und erstreckt sich von der Idee, dem Auftrag über die Entwicklung und Fertigung, die Auslieferung eines Produkts an den Endkunden bis hin zum Recycling einschließlich der damit verbundenen Diensleistungen.\grqq{}
\end{quotation}

Darüber, auf welcher Basis die digitale Transformation in der Industrie stattfinden wird, scheinen sich jedoch alle einig: durch die \textit{intelligente Vernetzung aller am Produktlebenszyklus beteiligten Menschen, Objekte und Systeme} \citep{Roth2016}. Das Wesentliche der Vernetzung bilden dezentrale \acf{cpss} \citep{Bendel2019a}. Die tatsächliche Wertschöpfung ergibt sich aus den in Echtzeit verfügbaren quantitativen Informationen, aus welchen man durch Analysen qualitative Erkenntnisse schließen und optimierte Aktionen auslösen kann \citep{Hnisch2017}. Die  Nach \cite{Sendler2016} gibt es für den Erfolg von \textit{Industrie 4.0} entscheidende Faktoren: Mittlerweile sind digitale Komponenten wie Sensoren, Aktoren oder Kameras so günstig und klein, dass sie in allen möglichen Bereichen eingebaut werden und Umgebungsdaten messen und aufnehmen können. Dank dem Internetprotokoll IPv6 und dem dadurch verfügbaren Adressraum können diese Komponenten ihren Platz im Internet finden. Dass die Informatik sich damit zur wichtigsten Ingenieursdisziplin entwickle, sei unterlässlich, da sie für die Vernetzung der Welt gebraucht werde.

\subsubsection{Historischer Kontext}

Um den aktuellen Stellenwert von Industrie 4.0 zu beschreiben, wird oft von der vierten industriellen \textit{Revolution} gesprochen. Revolutioniert wurde die Industrie erstmalig im 18. Jahrhundert mit der Erfindung der Dampfmaschine durch Thomas Newcomon und James Watt - die \textbf{erste industrielle Revolution} \citep{Roth2016}. Mit Errungenschaften wie dem dampfgetriebenen Webstuhl ging eine \textbf{Mechanisierung }der Produktion einher. Schon damals förderte eine Erfindung, die Lokomotive, eine Vernetzung, die einen regen Warenaustausch ermöglichte \citep{Barthelmaes2017}.
Durch die \textbf{Elektrifizierung} in der Industrie und der Zerlegung von Produktionsschritten in einzelne Einheiten konnten ab 1870 die Waren auf Fließ- und Förderbändern in Massen produziert werden. Angestoßen wurde die \textbf{zweite industrielle Revolution} von Erfindungen wie der Verbrennungskraftmaschine und dem Elektromotor sowie der Herstellung von Syntheseprodukten. Neben fossilen Energieträgern wie Kohle und Öl kam auch die Kernkraft hinzu \citep{Barthelmaes2017}.
Die \textbf{dritte industrielle Revolution} ab den 1970er Jahren, in der wir uns noch heute befinden,  brachte die \textbf{Automatisierung} der Produktion durch die \textbf{Digitalisierung} \citep{Voigt2018}. Getrieben wurde die Revolution durch das Wirtschaftswunder der 1960er Jahre \citep{Roth2016} und ermöglicht durch den Ausbau von Informations- und Kommunikationstechnologien. Entscheidende Technologien waren vielfältig. 1941 entwickelt der Bauingenieur Konrad Zuse den ersten programmgesteuerten und vollautomatischen Computer und setzte den Grundstein für eine rasante Entwicklung der nachfolgenden Technologien. Mit der Vebreitung von Mikroprozessoren, der Miniatisierung der Elektronik sowie der nach dem Mooreschen Gesetz vorausgesagten Zunahme der Prozessorstärke nahm die Welt ein neues Tempo an \citep{Sendler2016}. Einen nicht unwesentlichen Beitrag leistet die Raumfahrtechnik, ohne deren Satellitentechnik eine globale Kommunikation nicht möglich wäre. Da der energieintesive Einsatz dieser Technologien ein Bewusstsein über die Endlichkeit der fossilen Ressourcen schuf, kamen auch erneuerbare Energien hinzu \citep{Barthelmaes2017}.

Die Industrie 4.0 tatsächlich als \textbf{vierte Industrielle Revolution} zu bezeichnen wird kritisiert, da sie u.a. keine neuen technologischen Innovationen hervorbringe, sondern sich lediglich an den Technologien der dritten Revolution bediene \citep{Barthelmaes2017}. Auch wenn die Innovationen nicht unbedingt als revolutionär zu bezeichnen sind, findet durch Industrie 4.0 ein Wandel statt. Es entstehen eine Vielzahl neuer Geschäftsmodelle und Produktionsprozesse, die zu Effizienzsteigerungen führen \citep{Roth2016}.

\subsubsection{Technologische Treiber}
Hypebegriffe
Blockchain, Machine Learning, Big Data, Internet of Things, Ubiquitous Computing, Cloud Computing (kurz erwähnen und beschreiben)
\subsubsection{Kommunikationssysteme}
Kosten/Nutzen von Kommunikationssytemen
Metcalfe’s Law, Gilder’s Law, Moore’s Law
\begin{itemize}
  \item Kommunikationssprotokolle und Standards
  \item MQTT
  \item REST
  \item OPCUA
  \item etc
  \item nicht neue subsections sondern einfach Paragraph
\end{itemize}
